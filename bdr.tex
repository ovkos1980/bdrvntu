\documentclass{bdrvntu}

\usepackage{tabularx}    % для формування таблиць
\usepackage{amsmath}     % текст у формулах
\usepackage{amssymb}     %
\usepackage{minted}      % лістинги коду
\usepackage{listings}    %
\usepackage{lstfiracode} %
\usepackage{nicefrac}    % для дробів


 \usepackage{multirow}
% \usepackage{colortbl}
                                               
%!TEX root = ../bdr.tex
% основні параметри які можна змінити/задати в документі класу bdrvntu

%% навчальний заклад. По замовчанню "Вінницький національний технічний унверситет", але можна змінити на інший командою \educational{}, як наприклад:
%\educational{Національний університет <<Львівська політехніка>>}
%% навчальний заклад абревіатурою. По замовчанню "ВНТУ", але можна змінити на іншу командою \educationalabbr{}, як наприклад:
%\educationalabbr{НУЛП}
%% місто. По замовчанню "Вінниця", але можна змінити на інше командою \city{}, як наприклад:
%\city{Львів}
%% факультет. По замовчанню "комп'терних систем і автоматики", але можна змінити на інший командою \faculty{}, як наприклад:
\faculty{інтелектуальних інформаційних технологій та автоматизації}
%% кафедра. По замовчанню "автоматизації та інтелектуальних інформаційних технологій", але можна змінити на іншу командою \department{}, як наприклад:
%\department{метрології та промислової автоматики}
%% кафедра абревіатурою. По замовчанню "АІІТ", але можна змінити на іншу командою \departmentabbr{}, як наприклад:
%\departmentabbr{МПА}
%% тема роботи. Значення по замочанню не присвоєне, необхідно обов'язково задавати.
\title{вдосконалення системи управління якості програмного забезпечення}

%% студент
%% курсу. Значення по замочанню не присвоєне, необхідно обов'язково задавати.
\course{IV}
%% групи. Значення по замочанню не присвоєне, необхідно обов'язково задавати.
\group{АКІТ-20б}
%% спеціальності. Значення по замочанню не присвоєне, необхідно обов'язково задавати.
\speciality{151 -- Автоматизація та комп’ютерно-інтегровані технології}
%% ім'я. Значення по замочанню не присвоєне, необхідно обов'язково задавати.
\author{Петровський Д. Ю.}
%% ім'я в родовому відмінку. Фігурує в індивідуальному завданні. Значення по замочанню не присвоєне, необхідно обов'язково задавати.
\gcauthor{Петровському Дмитру Юрійовичу}
     
%% керівник
%% ім'я. Значення по замочанню не присвоєне, необхідно обов'язково задавати.
\leader{Овчинников К. В.}
%% науковий ступінь керівника. Значення по замочанню не присвоєне, необхідно обов'язково задавати.
\degree{к.т.н.}
%% педагогічне звання. Значення по замочанню не присвоєне, необхідно обов'язково задавати.
\position{доцент}
%% передбачається, що керівник з кафедри здобувача. Абревіатура для них однакова і задається командою \departmentabbr{}, що описани вище.

%% рецензент
%% ім'я. Значення по замочанню не присвоєне, необхідно обов'язково задавати.
\reviewer{Тарновський М. Г.}
% науковий ступінь рецензента. Значення по замочанню не присвоєне, необхідно обов'язково задавати.
\reviewerdegree{к.т.н.}
%% педагогічне звання. Значення по замочанню не присвоєне, необхідно обов'язково задавати.
\reviewerposition{доцент}
%% кафедра рецензента абревіатурою. Значення по замочанню не присвоєне, необхідно обов'язково задавати. 
\rdepartmentabbr{КСУ}

%% рік видання
%% рік видання проставляється автоматично в момент останньої компіляції, але якщо потрібно його можна змінити командою \annum{}, як наприклад:
% \annum{2021}

%% індивідуальне завдання
%% галузь знань. Значення по замочанню не присвоєне, необхідно обов'язково задавати.
\branchofknowledge{автоматизація та приладобудування} 
%% освітня програма. Значення по замочанню не присвоєне, необхідно обов'язково задавати.
\educationalprogram{автоматизація та комп’ютерно-інтегровані технології}

%% для креслеників
%% перевірив (нормоконтроль). Значення по замочанню не присвоєне, необхідно обов'язково задавати.
\controller{Овчинников К. В.}
%% затвердив. Значення по замочанню не присвоєне, необхідно обов'язково задавати.
\approver{Бісікало О. В.} 

%% статус очільника кафедри. Значення по замовчанню "Завідувач кафедри"
\dpheaderstatus{в.о. зав. кафедри}

\typeofwork{Бакалаврська дипломна робота}
\acstypeofwork{бакалаврську дипломну роботу}







  % варіативні значення 

\hypersetup{colorlinks,  % кольорові(сині) посилання по тексту
            allcolors=black,
            urlcolor=blue,
            pageanchor=false}

                                        
%\AtBeginEnvironment{thebibliography}{\interlinepenalty=10000}
                       
\begin{document}
\maketitle                           % титульний аркуш
%%!TEX root = ../bdr.tex
\begin{assignment}%
{}%наказ по університету
{}%строк подання роботи студентом
{}%вихідні дані до роботи
{}%зміст розрахунково-пояснювальної записки
{}%перелік графічного матеріалу
{}%консультанти розділів роботи
{}%дата видачі завдання
{}%календарний план
\end{assignment}    % порожнє індивідуальне завдання 
%!TEX root = ../bdr.tex
\begin{assignment}%
%наказ по університету
{\guillemotleft9\guillemotright~березня 2021 року №~65}
%
%строк подання роботи студентом
{07.06.2021}
%
%вихідні дані до роботи
{\begin{itemize}
\item {діапазон вимірювання тиску – не менше 70 МПа;}
\item {діапазон вимірювання переміщення – не менше 250 мм;}
\item {інтерфейс передавання даних – RS-485.}
\end{itemize}}
%
%зміст розрахунково-пояснювальної записки
{Провести огляд існуючих систем та засобів для визначення придатності до використання колісних пар залізничних вагонів, запропонувати структуру мікропроцесорної системи вихідного контролю пресових з’єднань колісних пар залізничних вагонів, розробити алгоритм роботи та програмне забезпечення мікропроцесорної системи вихідного контролю пресових з’єднань колісних пар залізничних вагонів}
%
%перелік графічного матеріалу
{Мікропроцесорна система вихідного контролю пресових з’єднань колісних пар заліз-ничних вагонів, схема електрична структурна; Мікропроцесорна система вихідно-го контролю пресових з’єднань колісних пар залізничних вагонів, схема електри-чна функціональна; Мікропроцесорна система вихідного контролю пресових з’єднань колісних пар залізничних вагонів, схема роботи програми}
%
%консультанти розділів роботи
{\consultant{1-3}{Овчинников К. В., доц. каф. АІІТ}
\consultant{4}{Овчинников К. В., доц. каф. АІІТ}}
%
%дата видачі завдання
{03.02.2021}
%
%календарний план
{\stage{1}{Вибір, узгодження та затвердження теми БДР}{03.09.2021}{06.03.2022}
\stage{2}{Аналіз літературних джерел. Попередня розробка основних розділів}{07.03.2022}{15.03.2022}
\stage{3}{Розробка технічного завдання (ТЗ)}{16.03.2022}{29.03.2022}
\stage{4}{Аналіз вирішення поставленої задачі}{30.03.2022}{19.04.2022}
\stage{5}{Розробка структурної схеми}{20.04.2022}{26.04.2022}
\stage{6}{Електричні розрахунки}{27.04.2022}{10.05.2022}
\stage{7}{Оформлення пояснювальної записки (ПЗ) та графічної частини}{25.05.2022}{01.06.2022}
\stage{8}{Перевірка ПЗ на відповідність вимогам}{02.06.2022}{07.06.2022}
\stage{9}{Попередній захист БДР, доопрацювання, рецензування БДР}{08.06.2022}{16.06.2022}
\stage{10}{Захист БДР ЕК}{20.06.2022}{20.06.2022}}
%
\end{assignment}           % індивідуальне завдання заповнене майже повністю 
%%!TEX root = ../bdr.tex
%two page annotations
\frontmatter
\chapter*{Анотація}% \chapter*{} - заголовок розділу без номера 

Бакалаврська дипломна робота присвячена пошуку шляхів вдосконалення системи управління якості програмного забезпечення. В роботі проводиться ґрунтовний огляд існуючих систем управління якості ПЗ, методів контролю якості та загальних підходів, що застосовуються для створення якісного програмного продукту. Визначається множина метрик для оцінки якості ПЗ на ранніх стадіях життєвого циклу.

\chapter*{Annotation}

The bachelor's thesis is devoted to finding ways to improve the software quality management system. The paper provides a thorough review of existing software quality management systems, quality control methods and general approaches used to create a quality software product. A set of metrics is determined to assess the quality of software in the early stages of the life cycle.

\mainmatter
\newpage      % анотації кожна на окремій сторінці
\input{main/annotation-sp.tex}       % дві анотації на одній сторінці
\tableofcontents                     % зміст                     
%!TEX root = ../bdr.tex
\chapter*{Вступ}

Сьогодні галузь розробки програмного забезпечення (ПЗ) вважається найбільшою галуззю світової економіки. За деякими оцінками на 2015 рік кількість розробників ПЗ у світі сягнула 19 мільйонів осіб, а до 2019 року прогнозується зростання їх кількості до 25 мільйонів. В Україні за попередніми оцінками на 2016 рік було налічено 99940 осіб зайнятих в IT галузі, і до цієї кількості входять не тільки програмісти, а й інженери з якості ПЗ, менеджери проектів, тощо. Інша ж частина людства безпосередньо залежить від їх успішної діяльності.

За час становлення галузі, завдання, їх складність, форми подання даних та методи їх обробки сильно змінились, але до сьогодні розробка якісного програмного забезпечення не стала нормою. Очевидно, що в галузі забезпечення якості ПЗ – криза. Великі проекти виконуються із запізненням або перевищенням кошторису витрат, розроблені програми не відповідають функціональним вимогам, ПЗ має низьку продуктивність, а якість програм не влаштовує користувачів.

Останнім часом група The Standish Group кожні два роки публікує звіт «The Chaos Report» з аналізом результатів впровадження програмних проектів. Результати аналізу показують, що у 2016 році лише 16,2 \% проектів завершились з дотриманням графіку та вклалися в кошторис витрат. Беручи до уваги той факт, що США щорічно витрачає більше ніж 250 млрд. дол. на IT галузь в цілому, на успішні проекти витрачається лише 40,5 млрд. дол. Інші кошти витрачаються на проекти, які ніколи не будуть впроваджені і не принесуть користі.

Наразі розроблено чимало засобів, методів, стандартів та технологій, які покликані забезпечити якість програмного забезпечення, але якість ПЗ, як і раніше, залежить в основному від знань та досвіду розробників. Проекти не рідко зазнають невдач через невірне формулювання вимог, невдале проектування або неефективне планування, неправильне розуміння або недостатній аналіз специфікації, і, як наслідок, помилки на ранніх етапах життєвого циклу програмного забезпечення. Тому, виявляється, що кращий спосіб підвищення якості ПЗ є мінімізація часу, який витрачається на виправлення коду, в наслідок зміни вимог, зміни проекту чи виконання відлагоджування. Це підтверджується реальними даними досліджень проведених в лабораторіях NASA, які показали, що підвищена увага до конт-ролю якості дозволяла знизити рівень помилок, але не підвищувала загальні витрати на розробку.

Проте процес розробки програмного забезпечення, як і процес оцінювання якості ПЗ залишаються такими, що вимагають системного підходу, що має наукове підґрунтя. Тому пошук шляхів покращення системи управління якістю програмного забезпечення залишається актуальною задачею.
Мета і задачі дослідження. Метою роботи є пошук шляхів вдосконалення системи управління якості програмного забезпечення в рамках існуючих методик та підходів до оцінки якості.

Для досягнення мети сформульовано наступні задачі:
\begin{itemize}
\item	провести огляд існуючих систем якості;
\item	проаналізувати методики та підходи до оцінки якості програмного забезпечення;
\item	визначити найбільш ефективні, визначити недоліки, запропонувати шляхи покращення;
\item	підтвердити справедливість зроблених припущень, сформулювати задачі подальшого дослідження.
\end{itemize}

\textit{Об’єктом} роботи є система управління якості програмного забезпечення.

\textit{Предметом} роботи є процес вдосконалення існуючої системи управління якості програмного забезпечення, підвищення ефективності методів та підходів до визначення якості ПЗ.

\textit{Практичне значення} одержаних в роботі результатів полягає в тому, що:
\begin{enumerate}
\item	в результаті аналізу існуючих методів і підходів визначені найбільш слабкі місця в сучасній системі управління якістю програмного забезпечення;
\item	запропонована множина метрик якості та діапазони їх зміни для контролю якості ПЗ на ранніх етапах життєвого циклу.
\end{enumerate}
             % вступ
%!TEX root = ../bdr.tex
\chapter{Верстування текстової частини роботи}
А це просто текст...

Порожній рядок означає, що це був абзац

%\textbf{Lorem Ipsum is simply dummyl}  text of the printing and typesetting industry. Lorem Ipsum has been the industry's standard dummy text ever since the 1500s, when an unknown printer took a galley of type and scrambled it to make a type specimen book. It has survived not only five centuries, but also the leap into electronic typesetting, remaining essentially unchanged. It was popularised in the 1960s with the release of Letraset sheets containing Lorem Ipsum passages, and more recently with desktop publishing software like Aldus PageMaker including versions of Lorem Ipsum. 
\lipsum[2-4]

А це ненумерований список:
\begin{itemize}
\item Зелений;
\item Жовтий;
\item Білий.
\end{itemize}

А це нумерований список:
\begin{enumerate}
\item Колір
      \begin{enumerate}
      \item Зелений;
      \item Жовтий;
      \item Білий.
      \end{enumerate}	
\item Температура
      \begin{enumerate}
      \item Теплий;
      \item Холодний.
      \end{enumerate}
\end{enumerate}
 

\section{Рисунки}

Вставка рисунків робиться дуже просто:

\begin{figure}[h]
\caption{Опис рисунку}
\end{figure}

Далі можна вставити і сам рисунок (рис. \ref{fig:tux}). Він має бути у графічному файлі ({*}.jpg, {*}.png тощо). 

\begin{figure}[h]
 \centering\includegraphics{img/Tux.png}
 \caption{Пінгвін}
 \label{fig:tux}
\end{figure}

Рисунок можна винести і в додаток (додаток \ref{apdx:text}), \label{linkpage} а посилання на нього залишити в основній частині роботи (рис. \ref{apdxfig:tux}).
Можна вручную вставляти картинки і прописувати кожний параметр.
Параметр width визначає ширину рисунку. У цьому випадку вона дорівнюватиме ширині текста \verb|\textwidth|. Замість \verb|\textwidth| можна вказати значення від 0.1 до 1. \verb|\label| потрібна для того, щоби потім можна було послатись на цю картинку, як тут.

\begin{figure}[!htb]
 \centering
 \includegraphics[width=0.8\textwidth]{img/fig1.png}
 \caption{Використовуйте Rich Text}
 \label{fig:fig1}
\end{figure}

\section{Таблиці}
Таблиці  - це складний об'єкт. Тому всі параметри треба прописувати вручну

\begin{table}[h!]
\centering
\begin{tabular}{|c|c|c|c|} 
 \hline
 Стопець1 & Стопець2 & Стопець3 & Стопець4 \\ [0.5ex] 
 \hline
 \multirow{3}{5em}{Декілька рядків} & 6 & 87837 & 787 \\ 
  &  7 & 78 & 5415 \\
   & 545 & 778 & 7507 \\
   & 545 & 18744 & 7560 \\
   & 88 & 788 & 6344 \\ [1ex] 
 \hline
\end{tabular}
\caption{Приклад роботи з таблицею}
\label{table:1}
\end{table}

\begin{table}[h]
	\caption{\label{table:2}Функціональна залежність параметрів ...}
	%\label{table:2}
	\begin{tabular}{|c|c|c|}
		\hline 
		Індекс & Показник 1 & Показник 2\tabularnewline
		\hline 
		\hline 
		1 & 2 & 3\tabularnewline
		\hline 
		4 & 5 & 6\tabularnewline
		\hline 
	\end{tabular}
\end{table}

Для табл. \ref{table:2} вже можете побачити легенду, яка на відміну від рисунків розташована зверху.

Великі таблиці можна повертати на 90 градусів:

%\begin{sidewaystable}
%	\centering{}
%	\caption{Набір компонентів.}
%	\begin{tabular}{|c|>{\raggedright}p{0.25\columnwidth}|>{\centering}p{0.15\columnwidth}|>{\centering}p{0.1\columnwidth}|}
%	\hline 
%	№ & Компонент  & Маса & К-сть\tabularnewline
%	\hline 
%	\hline 
%	1 & Компонент А & 12.8 & 100\tabularnewline
%	\hline 
%	2 & Компонент Б & 7.88 & 12\tabularnewline
%	\hline 
%	3 & Компонент В & 83.89 & 44\tabularnewline
%	\hline 
%	\end{tabular}
%\end{sidewaystable}

Більше про таблиці можна прочитати \href{https://www.overleaf.com/learn/latex/Tables}{тут}.

%\clearpage{}

\section{Формули...}

Формули, що входять до бакалврської роботи, нумерують в межах розділу. Номер формули складається з номера розділу та порядкового номера формули, розділених крапкою. Номер формули розташовують з правого боку на рівні формули в круглих дужках. По\-си\-лан\-ня в тексті на номер формули дають в дужках, наприклад, «... за формулою (\ref{eq:explan})». За необхідності вказують одиницю вимірювання, беручи її в квадратні дужки

\begin{equation}
\label{eq:explan}
I = \frac{U}{R}~[A].
\end{equation}

Числову підстановку і розрахунок виконують з нового рядка не нумеруючи. Одиницю вимірювання беруть в круглі дужки. Наприклад,

$$I = \frac{220}{100}~(\text{А}).$$

Розмірність одного й того ж параметра в межах документа має бути однаковою. Якщо формула велика, то її можна переносити в на\-ступ\-ні рядки. Перенесення виконують тільки математичними знаками, повторюючи знак на початку наступного рядка. При цьому знак мно\-жен\-ня «$\cdot$» замінюють знаком «$\times$».


\begin{align}
\label{xt:eq}
y(t) &= \frac{1}{{\rho}\,{S}\,{C_{f}}\,\sin\alpha}\Big(\ln\Big(\frac{1}{1962\,m}\Big(10\,v_0\sqrt{\rho\,S\,C_f\,\sin^3\alpha}\,\times \nonumber\\  &\times\cos\Big(\frac{3\,t\,\sqrt{218\,\rho\,S\,C_f\,\sin\alpha}}{20\,\sqrt{m}}\Big)\Big)^2\Big)\,m\Big).
\end{align}


Пояснення символів та числових коефіцієнтів наводять під формулою. Пояснення кожного символа подається з нового рядка в тій послідовності, в якій символи зустрічаються в формулі. Перший рядок пояснення починається зі слова «де» без двокрапки після нього.

\begin{equation}
T = 2\pi\sqrt{\frac{m}{k}}, 
\end{equation}

\begin{explanation}
\fitem $k$ -- коефіцієнт жорсткості пружини; 
\item $m$ -- маса тягарця.
\end{explanation}

Формула є частиною речення, тому до неї застосовують такі ж правила граматики, як і до інших членів речення. Якщо формула зна\-хо\-дить\-ся в кінці речення, то після неї ставлять крапку. 

Вставляти в текст можна формули будь-якої складності, як наприклад:

\begin{equation}
	\text{f=}\sqrt[12]{\sum\left(\beta*\xi\right)*\frac{z}{\arctan(123)}}*\top f(z)
\end{equation}


Формули з нумерацією    
\begin{equation}
    f(x)=\frac{x}{1+x^2}
\end{equation}

Ще одна формула    
\begin{equation}
    x^n + y^n = z^n
\end{equation}
    
А так прописуються формули, де нумерація не потрібна \(x^2 + y^2 = z^2\).

Другий спосіб - це вставити формулу між символам долара: $x^2 + y^2 = z^2$. 

\section{Формули без нумерації. Дроби.}
Для відтворення дробей в рядку, наприклад \(\frac{3x}{2}\), можна встановити інший стиль:
    \( \displaystyle \frac{3x}{2} \).
Це також працює і у зворотньому напрямку:
    \[ f(x)=\frac{P(x)}{Q(x)} \ \ \textrm{та}
    \ \ f(x)=\textstyle\frac{P(x)}{Q(x)} \]

\section{Інтеграли}
Інтеграл \(\int_{a}^{b} x^2 dx\) всередені тексту.
    \medskip
І той самий інтеграл між рядками:
    \[
    \int_{a}^{b} x^2 \,dx
    \]
    
Детальніше можна прочитати 
\href{https://www.overleaf.com/learn/latex/Integrals,_sums_and_limits#Integrals}{тут}.

\section{Суми і добутки}
Читати 
\href{https://www.overleaf.com/learn/latex/Integrals,_sums_and_limits#Sums_and_products}{тут}.

\section{Межі}
    
    Межа \(\lim_{x\to\infty} f(x)\) всередені тексту.
    І між рядками:
    \[
    \lim_{x\to\infty} f(x)
    \]

\section{Символи}
$\alpha A$ - грецькі,  $ \lambda; \Lambda$ - физичні величины=и, $\exists; \forall$ - логічні символм\\
За \href{https://www.overleaf.com/learn/latex/List_of_Greek_letters_and_math_symbols}{цим посиланням} можна одержати інформацію про інші символи. 
А \href{https://www.overleaf.com/learn/latex/Operators}{тут} - математичні оператори.

\section{Зноски, примітки та інш.}
У тексті зручно робити різноманітні зноски та примітки. Для цього їх достатньо встановити у необхідному місці, а форматування зробить свою потужну роботу. Ось приклад зноски, що розташується знизу \footnote{Це тест до зноски 1}. 

\section{LaTeX Wiki}
 Повне введення у LaTeX читайти \href{https://www.texlive.info/CTAN/info/lshort/russian/lshortru.pdf}{тут}.




            % перший розділ
%!TEX root = ../bdr.tex

\chapter{Створення бібліоґрафії}

У TeХ \emph{бібліографія створюється автоматично}. Для цього у необхідному місці тексту достатньо вставити команду \verb|\bibliography{bibdata}| і аргументом вказати ім'я файлу із літературними джерелами.
Цей файл повинен мати розширення \verb|*.bib| (наприклад, для цього шаблону використовується файл \emph{bibdata.bib}). 

У цьому файлі знаходиться \emph{простий невпорядкований список} записів кожен з яких
описує рівно одну публікацію (тип видання, назва, автор, рік, сторінки тощо). Редагувати цей файл можна або спеціальною програмою (наприклад, JabRef) - і це зручно, або навіть будь-яким текстовим редактором. 

Власне оформлення самої бібліографії здійснюється повністю автоматично. При цьому відбувається впорядкування, нумерація і форматувння літературних джерел згідно правил. 

У бібліографію потрапляють не всі літературні джерела із файлу \verb|*.bib|, а лише ті, на які було здійснено посилання із тексту. Це означає, що можна вести один універсальний файл (базу даних) з літературою за певною тематикою, а у статтях посилатись лише на необхідні.

\section{Посилання на літературу}

Якщо існує bib-файл і він підключений до команди ``Бібліографія'', то у тексті можна робити посилання на літературні джерела, наприклад, ось так \cite{WinNT} або так \cite{Vasylenko92, Afanasyev92} або так \cite{Makilov91, Ponomarenko86, Belousova81, Tezisy, Statia, GOST7184}.

Нумерація літератури у тексті і у бібліографії здійснюється автоматично. 

            % другий розділ
%!TEX root = ../bdr.tex
\chapter*{Висновки}

Аналіз проведений в роботі вказує на те, що сьогодні існує велика кількість стандартів та технологій покликаних підвищувати якість програмного забезпе-чення, проте якість готової продукції як і раніше залишається на достатньо низь-кому рівні. Достатньо лише того, що лише 16 \% програмних проектів розпочатих в попередніх роках завершились з дотриманням графіку та перейшли у фазу заве-ршеного програмного продукту.

Очевидно, що існуючі підходи при всій своїй повноті та рівні автоматизації не здатні забезпечити достатній рівень якості ПЗ. Аналіз показав, що не малу час-тину контролю для забезпечення рівня якості необхідно проводити на ранніх ета-пах життєвого циклу програмного забезпечення.

В результаті дослідження були визначені метрики якості, та визначена їх ефективність. Обраховані граничні значенні та діапазони зміни спираючись на дані проведених досліджень. Серед найбільш ефективних метрик якості придат-них до застосування на ранніх етапах життєвого циклу ПЗ були обрані: метрика звертання до глобальних змінних, кількість виявлених помилок при інспектуван-ні, цикломатична складність, відносна гранична складність програми. Результати досліджень показують, що на рівень якості впливають не тільки метрики з широ-ким діапазоном зміни значень і нехтувати менш значущими метриками не доціль-но.
             % висновки
%!TEX root = ../bdr.tex

%ГОСТ 7.1 2003
%\bibliographystyle{gost2003vntu}

%ДСТУ 8302:2015
\bibliographystyle{gost2008vntu}
\def\BibEmph#1{\emph{#1}}    % для краси автори будуть друкуватись курсивом
\def\BibDash{}               % забираємо зайві тире

%prevent the last bibliography entry from being split in a new page. "etoolbox" package need to be loaded
%\AtBeginEnvironment{thebibliography}{\interlinepenalty=10000}

\bibliography{bibdata}               % перелік послань
\appendix                            % додатки
%!TEX root = ../bdr.tex
\chapter[(Довідковий) Приклад текстового додатку]{(Довідковий)\\ Приклад текстового додатку}
\label{apdx:text}

Додатки створюються так само, як і основний текст пояснювальної записки. Будь-який додаток -- це окремий розділ, який
починається командою \verb|\chapter{}| але, на відміну від розділів, що розташовуються в основній частині роботи, заголовок
додатку буде формуватись зі слова <<Додаток>>  і літери алфавіту розташованих згори по центру автоматично. Таку зміну
стандартної поведінки команди \verb|\chapter{}| провокує команда \verb|\appendix|, яка розбиває документ на дві частини: 
основну частину та частину з додатками.


Як і в основній частині роботи в додатках можна розташовувати рисунки, проте нумеруватись вони будуть в межах додатку
починаючись літерою, як показано на рисунку \ref{apdxfig:tux}. Посилання на рисунки розташовані в додатках можна розташовувати 
в будь-якому місці документу (дивись сторінку \pageref{linkpage}).


\begin{figure}[h]
 \centering\includegraphics{img/Tux.png}
 \caption{Пінгвін}
 \label{apdxfig:tux}
\end{figure}

Все вище описане можна застосувати і для таблиць. Як і рисунки таблиці будуть нумеруватись в межах додатку починаючи літерою.
Посилатись на таблиці розташовані в додатку можна так само як і на будь-які інші таблиці в документі. 

\begin{table}[h]
\caption{\label{apdxtable:1}Функціональна залежність параметрів ...}
 \begin{tabular}{|c|c|c|}
 \hline 
 Індекс & Показник 1 & Показник 2\tabularnewline
 \hline\hline 
 1      & 2          & 3         \tabularnewline
 \hline
 4      & 5          & 6         \tabularnewline
 \hline 
\end{tabular}
\end{table}
  


\begin{table}[h]
\underonespace
\caption{\label{apdxtable:2}Основні та деякі похідні одиниці системи SI}
\begin{tabular}{|>{\raggedright}m{5cm}|c|m{4cm}|c|}
\hline
\multicolumn{2}{|c|}{Величина}&\multicolumn{2}{c|}{Одиниця}\\[0pt]\hline\vspace{4pt}
Найменування & Розмірність & Найменування & Позначення \\[0pt]\hline\hline\vspace{4pt}
Довжина & $L$ & метр & м \\[0pt]\hline\vspace{4pt}
Маса & $M$ & кілограм & кг \\[0pt]\hline\vspace{4pt}
Час & $T$ & секунда & с \\[0pt]\hline\vspace{4pt}
Сила електричного струму & $I$ & ампер & А \\[0pt]\hline\vspace{4pt}
Термодинамічна температура & $\Theta$ & кельвін & К \\[0pt]\hline\vspace{4pt}
Кількість речовини & $N$ & моль & моль \\[0pt]\hline\vspace{4pt}
Сила світла & $J$ & кандела & кд \\[0pt]\hline
\hline\vspace{4pt}
Плаский кут & $1$ & радіан & рад \\[0pt]\hline\vspace{4pt}
Площа & $L^2$ & квадратний метр & м$^2$ \\[0pt]\hline\vspace{4pt}
Об'єм & $L^3$ & кубічний метр & м$^2$\\[0pt]\hline\vspace{4pt}
Швидкість & $LT^{-1}$ & метр за секунду & м/c \\[0pt]\hline\vspace{4pt}
Прискорення & $LT^{-2}$ & метр за секунду в квадраті & м/с$^2$\\[0pt]\hline\vspace{4pt}
Кутова швидкість & $T^{-1}$ & радіан за секунду & рад/с \\[0pt]\hline\vspace{4pt}
Кутове прискорення & $T^{-2}$ & радіан за секунду в квадраті & рад/с$^2$ \\[0pt]\hline\vspace{4pt}
Щільність & $ML^{-3}$ & кілограм на метр кубічний & кг/м$^3$ \\[0pt]\hline\vspace{4pt}
Питомий об'єм & $L^3M^{-1}$ & кубічний метр на кілограм & м$^3$/кг \\[0pt]\hline\vspace{4pt}
Сила & $LMT^{-2}$ & ньютон & Н \\[0pt]\hline\vspace{4pt}
Тиск & $L^{-1}MT^{-2}$ & паскаль & Па \\[0pt]\hline\vspace{4pt}
Температура Цельсія & $\Theta$ & градус Цельсія & $^\circ$C \\[0pt]\hline 
\end{tabular}
\end{table}       % додаток перший
%!TEX root = ../bdr.tex
\begin{a3paperl}
\chapter[(Довідковий) Приклад великоформатного додатку]{Приклад великоформатного додатку}\label{apdx:a3}

Якщо ілюстративний матеріал для відображення вимагає більшого ніж А4 формату в класі \verb|bdrvntu| передбачені
оточення для інших форматів аркушу в довільній орієнтації. Даний додаток офрмлений на аркуші А3 формату в альбомній орієнтації.
Для креслеників в класі \verb|bdrvntu| передбачено оточення \verb|drawing| для формування стандартних  штампів на стандартних
форматах аркушів (дивись додаток \ref{apdx:ozpsch}).  

\begin{figure}[h]
 \centering\includegraphics{img/umldiagram.png}
 \caption{UML-діаграма}
 \label{apdxfig:umldiagram}
\end{figure}

\end{a3paperl}

        % додаток другий
%!TEX root = ../bdr.tex

\acode{08-02}%
\bcode{БКР}%
\ccode{000}%
\dcode{13}%
\ecode{000}
\fcode{Е3}
\partdes{Схема електрична принципова}%

\begin{drawing}
\chapter[(Довідковий) Приклад додатку у форматі кресленика]{}
 \begin{picture}(0,0)
  \put(-70,-750){\hbox{\includegraphics{img/sch4.pdf}}}
 \end{picture}
\label{apdx:ozpsch}
\end{drawing}

     % додаток третій
%!TEX root = ../bdr.tex
\chapter[(Довідковий) Приклад додатку з лістингом]{Приклад додатку з лістингом}

%\begin{minted}  % приклад вставки коду вручну через середовище minted 
%[
%frame=lines,
%framesep=2mm,
%baselinestretch=1.2,
%fontsize=\footnotesize,
%linenos
%]
%{python}
%import numpy as np
%    
%def incmatrix(genl1,genl2):
%    m = len(genl1)
%    n = len(genl2)
%    M = None #to become the incidence matrix
%    VT = np.zeros((n*m,1), int)  #dummy variable
%    
%    #compute the bitwise xor matrix
%    M1 = bitxormatrix(genl1)
%    M2 = np.triu(bitxormatrix(genl2),1) 
%
%    for i in range(m-1):
%        for j in range(i+1, m):
%            [r,c] = np.where(M2 == M1[i,j])
%            for k in range(len(r)):
%                VT[(i)*n + r[k]] = 1;
%                VT[(i)*n + c[k]] = 1;
%                VT[(j)*n + r[k]] = 1;
%                VT[(j)*n + c[k]] = 1;
%                
%                if M is None:
%                    M = np.copy(VT)
%                else:
%                    M = np.concatenate((M, VT), 1)
%                
%                VT = np.zeros((n*m,1), int)
%    
%    return M
%\end{minted}

{\underonespace
\begin{verbatim}
#include <cstdlib>
#include <iostream>

struct node_t
 {
   int data;
   node_t *next;
   node_t *prev;    
 };

node_t *head = NULL, *tail = NULL; 

int add_head(int);
int add_tail(int);
int size_list(void);

int main(int argc, char *argv[])
{
    add_head(1);
    printf("%d\n", size_list());
    add_tail(2);
    printf("%d\n", size_list());
    system("PAUSE");
    return EXIT_SUCCESS;
}

int add_head(int d)
 {
   node_t *tmp = new node_t;
   if(tmp!=NULL)
    {
      tmp->data = d;
      tmp->next = head;
      tmp->prev = NULL;
      if(head!=NULL) head->prev = tmp;
       else if(tail==NULL) tail = tmp;
             else return -1;
      head = tmp;
      return 0;          
    } else return -1;
 }

\end{verbatim}}

\clearpage

{\underonespace
\begin{verbatim}
int add_tail(int d)
 {
   node_t *tmp = new node_t;
   if(tmp!=NULL)
    {
      tmp->data = d;
      tmp->next = NULL;
      tmp->prev = tail;
      if(tail!=NULL) tail->next = tmp;
       else if(head==NULL) head = tmp;
             else return -1;
      tail = tmp;
      return 0;          
    } else return -1;             
 }

int size_list(void)
 {
   if(head!=NULL)
    {
      int num = 0;           
      node_t *tmp = head;
      while(tmp!=NULL)
       {
         tmp = tmp->next;
         num++;            
       }
      return num;            
    } else return 0;
 }
\end{verbatim}}
    % додаток четвертий
\end{document}
