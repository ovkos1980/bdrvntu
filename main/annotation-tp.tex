%!TEX root = ../bdr.tex
%two page annotations
\frontmatter
\chapter*{\abstractname}% \chapter*{} - заголовок розділу без номера 
\thispagestyle{empty}

Поданий документ складається зі вступу, двох розділів, висновків, переліку використаних джерел
(\total{citnum} бібліографічних посилань) та додатків. Загальний обсяг роботи (\total{page} сторінки) містить
рисунків \total{figures}, таблиць \total{tables}.

Шаблон створено для ознайомлення з основними можливостями видавничої системи {\LaTeX} в цілому та можливостями 
класу документу \verb|bdrvntu| зокрема. Автори в доступній формі спробували донести до читача основні принципи
створення документу.

Ключові слова: видавнича система, {\TeX}, {\LaTeX}, бакалавр, кваліфікаційна робота.

\chapter*{Annotation}
\thispagestyle{empty}

The submitted document consists of an introduction, two sections, conclusions,
list of used sources (\total{citnum} bibliographic references) and appendices.
The total volume of work (\total{page} pages) contains figures \total{figures}, tables \total{tables}.

The template was created to get acquainted with the main features of
the old {\LaTeX} system as a whole and the capabilities of the bdrvntu document class
in particular. The authors tried to convey the main points to the reader in an accessible form
principles of document creation.

Key words: publishing system, {\TeX}, {\LaTeX}, bachelor, qualification
valuable work.

\newpage
\mainmatter