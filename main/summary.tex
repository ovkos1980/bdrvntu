%!TEX root = ../bdr.tex
\chapter*{Висновки}

Аналіз проведений в роботі вказує на те, що сьогодні існує велика кількість стандартів та технологій покликаних підвищувати якість програмного забезпе-чення, проте якість готової продукції як і раніше залишається на достатньо низь-кому рівні. Достатньо лише того, що лише 16 \% програмних проектів розпочатих в попередніх роках завершились з дотриманням графіку та перейшли у фазу заве-ршеного програмного продукту.

Очевидно, що існуючі підходи при всій своїй повноті та рівні автоматизації не здатні забезпечити достатній рівень якості ПЗ. Аналіз показав, що не малу час-тину контролю для забезпечення рівня якості необхідно проводити на ранніх ета-пах життєвого циклу програмного забезпечення.

В результаті дослідження були визначені метрики якості, та визначена їх ефективність. Обраховані граничні значенні та діапазони зміни спираючись на дані проведених досліджень. Серед найбільш ефективних метрик якості придат-них до застосування на ранніх етапах життєвого циклу ПЗ були обрані: метрика звертання до глобальних змінних, кількість виявлених помилок при інспектуван-ні, цикломатична складність, відносна гранична складність програми. Результати досліджень показують, що на рівень якості впливають не тільки метрики з широ-ким діапазоном зміни значень і нехтувати менш значущими метриками не доціль-но.
