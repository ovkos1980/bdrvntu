%!TEX root = ../bdr.tex
\begin{assignment}%
%наказ по університету
{\guillemotleft9\guillemotright~березня 2021 року № 65}
%
%строк подання роботи студентом
{07.06.2021}
%
%вихідні дані до роботи
{\begin{itemize}
\item {діапазон вимірювання тиску – не менше 70 МПа;}
\item {діапазон вимірювання переміщення – не менше 250 мм;}
\item {інтерфейс передавання даних – RS-485.}
\end{itemize}}
%
%зміст розрахунково-пояснювальної записки
{Провести огляд існуючих систем та засобів для визначення придатності до використання колісних пар залізничних вагонів, запропонувати структуру мікропроцесорної системи вихідного контролю пресових з’єднань колісних пар залізничних вагонів, розробити алгоритм роботи та програмне забезпечення мікропроцесорної системи вихідного контролю пресових з’єднань колісних пар залізничних вагонів}
%
%перелік графічного матеріалу
{Мікропроцесорна система вихідного контролю пресових з’єднань колісних пар заліз-ничних вагонів, схема електрична структурна; Мікропроцесорна система вихідно-го контролю пресових з’єднань колісних пар залізничних вагонів, схема електри-чна функціональна; Мікропроцесорна система вихідного контролю пресових з’єднань колісних пар залізничних вагонів, схема роботи програми}
%
%консультанти розділів роботи
{\begin{tabular}{|p{1.8cm}|p{8.4cm}|p{2.5cm}|p{2.5cm}|}\hline
  \multirow{2}{*}{Розділ} & \multirow{2}{*}{\parbox{8.4cm}{\centering{Прізвище, ініціали та посада\\ консультанта}}} & \multicolumn{2}{c|}{Підпис, дата} \\ \cline{3-4}
  & & завдання видав & завдання прийняв \\ \hline
  1 & ??? &  &  \\ \hline 
  2 & ??? &  &  \\ \hline
  3 & ??? &  &  \\ \hline
  4 & ??? &  &  \\ \hline
\end{tabular}}
%
%дата видачі завдання
{03.02.2021}
%
%календарний план
{\begin{tabular}{|p{0.8cm}|p{9.0cm}|p{3.4cm}|p{2.0cm}|}\hline
  № з/п & Назва етапів бакалаврської дипломної роботи & Строк виконання & Примітка \\ \hline
  1 & Огляд існуючих систем та засобів визначення придатності  пресових з’єднань & 01.02.2021 26.02.2021 &  \\ \hline 
  2 & Огляд методів вимірювання надлишкового тиску та переміщення &  &  \\ \hline
  3 & Розробка схеми функціональної мікропроцесорної системи &  &  \\ \hline
  4 & Вибір компонентів для використання в складі мікропроцесорної  системи &  &  \\ \hline
\end{tabular}}
\end{assignment} 