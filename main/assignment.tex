%!TEX root = ../bdr.tex
\begin{assignment}%
%наказ по університету
{\guillemotleft9\guillemotright~березня 2021 року №~65}
%
%строк подання роботи студентом
{07.06.2021}
%
%вихідні дані до роботи
{\begin{itemize}
\item {діапазон вимірювання тиску – не менше 70 МПа;}
\item {діапазон вимірювання переміщення – не менше 250 мм;}
\item {інтерфейс передавання даних – RS-485.}
\end{itemize}}
%
%зміст розрахунково-пояснювальної записки
{Провести огляд існуючих систем та засобів для визначення придатності до використання колісних пар залізничних вагонів, запропонувати структуру мікропроцесорної системи вихідного контролю пресових з’єднань колісних пар залізничних вагонів, розробити алгоритм роботи та програмне забезпечення мікропроцесорної системи вихідного контролю пресових з’єднань колісних пар залізничних вагонів}
%
%перелік графічного матеріалу
{Мікропроцесорна система вихідного контролю пресових з’єднань колісних пар заліз-ничних вагонів, схема електрична структурна; Мікропроцесорна система вихідно-го контролю пресових з’єднань колісних пар залізничних вагонів, схема електри-чна функціональна; Мікропроцесорна система вихідного контролю пресових з’єднань колісних пар залізничних вагонів, схема роботи програми}
%
%консультанти розділів роботи
{\begin{tabular}{|c|p{10.1cm}|c|c|} 
\hline
  \multirow{3}{*}{Розділ} 
& \multirow{3}{*}{\begin{tabular}[c]{@{}l@{}}Прізвище, ініціали та посада  \\консультанта\end{tabular}} 
& \multicolumn{2}{c|}{Підпис, дата} \\ 
\cline{3-4}
& & завдання & завдання \\
& & видав    & прийняв  \\ 
\hline\hline
& & & \\
1-4 & Овчинников К. В., доц. каф. АІІТ &  & \\
& & & \\
\hline
\end{tabular}}
%
%дата видачі завдання
{03.02.2021}
%
%календарний план
{\raggedright
\begin{tabular}{|c|m{7.7cm}|c|c|c|} 
\hline
  \multirow{2}{*}{\begin{tabular}[c]{@{}c@{}}№ \\з/п \end{tabular}} 
& \multirow{2}{*}{\begin{tabular}[c]{@{}l@{}}Назва етапів бакалаврської  \\дипломної роботи \end{tabular}} 
& \multicolumn{2}{c|}{Термін виконання} 
& \multirow{2}{*}{\begin{tabular}[c]{@{}l@{}}Примітка\\ \end{tabular}} \\ 
\cline{3-4} &  & початок & закінчення & ~ \\ 
\hline\hline
1 & Вибір, узгодження та затвердження теми БДР & 03.09.2021 & 06.03.2022 & ~ \\ 
\hline
2 & Аналіз літературних джерел. Попередня розробка основних розділів & 07.03.2022 & 15.03.2022 & ~ \\ 
\hline
3 & Розробка технічного завдання (ТЗ) & 16.03.2022 & 29.03.2022 & ~ \\ 
\hline
4 & Аналіз вирішення поставленої задачі & 30.03.2022 & 19.04.2022 & ~ \\ 
\hline
5 & Розробка структурної схеми & 20.04.2022 & 26.04.2022 & ~ \\ 
\hline
6 & Електричні розрахунки & 27.04.2022 & 10.05.2022 & ~ \\ 
\hline
7 & Оформлення пояснювальної записки (ПЗ) та графічної частини & 25.05.2022 & 01.06.2022 & ~ \\ 
\hline
8 & Перевірка ПЗ на відповідність вимогам & 02.06.2022 & 07.06.2022 & ~ \\ 
\hline
9 & Попередній захист БДР, доопрацювання, рецензування БДР & 08.06.2022 & 16.06.2022 & ~ \\ 
\hline
10 & Захист БДР ЕК & 20.06.2022 & 20.06.2022 & ~ \\
\hline
\end{tabular}}
\end{assignment} 