%!TEX root = ../bdr.tex
\chapter[(Довідковий) Приклад текстового додатку]{Приклад текстового додатку}\label{apdx:tasks}

Провести математичне моделювання процесу руху тіла в просторі для чого:
\begin{enumerate}
\item описати фізичну модель, проаналізувати параметри та діапазон їх зміни. Зробити необхідні припущення та спрощення;
\item побудувати математичну модель;
\item розробити програму для дослідження параметрів моделі;
\item провести обчислювальний експеримент, отримати шукані величини та представити результати моделювання.
\end{enumerate}

\vspace{1em}
\textit{Варіант 1}\par
Абсолютна величина вектору початкової швидкості кульки, що кинута під кутом 30 градусів до горизонту дорівнює 40 м/с. Визначте дальність польоту кульки. Опором повітря знехтувати.

\vspace{1em}
\textit{Варіант 2}\par
Снаряд вилітає зі зброї з початковою швидкістю 490 м/с під кутом 30 градусів до горизонту. Знайти висоту, дальність і час польоту снаряда. Опором повітря нехтувати.

\vspace{1em}
\textit{Варіант 3}\par
З вежі кинуто тіло в горизонтальному напрямі із швидкістю 40 м/с. Яку швидкість набуде тіло через 3 с після початку руху? Опором повітря знехтувати.

\vspace{1em}
\textit{Варіант 4}\par
Тіло кинуте під кутом 45 градусів до поверхні землі. Визначити дальність польоту, якщо початкова швидкість дорівнює 60 м/с. Опором повітря знехтувати.

\vspace{1em}
\textit{Варіант 5}\par
З якою швидкістю потрібно кинути кульку, щоб вона перелетіла огорожу, якщо кут кидання дорівнює 60 градусів, висота огорожі 5 м, відстань між кулькою і огорожею 7 м. Опором повітря знехтувати.

\vspace{1em}
\textit{Варіант 6}\par
Під яким кутом потрібно кинути камінь, щоб він потрапив в лунку, якщо абсолютна величина вектору початкової швидкості дорівнює 50 м/с, відстань між каменем і лункою 20 м. Опором повітря знехтувати.

\vspace{1em}
\textit{Варіант 7}\par
Абсолютна величина вектору початкової швидкості кульки, що кинута під кутом 30 градусів до горизонту дорівнює 40 м/с, маса кульки – 1 кг, коефіцієнт опору – 0,3 Визначите дальність польоту кульки (з урахуванням опору повітря).

\vspace{1em}
\textit{Варіант 8}\par
Снаряд вилітає зі зброї з початковою швидкістю 490 м/с під кутом 30 градусів до горизонту, маса снаряда – 10 кг, коефіцієнт тертя – 0,2. Знайти висоту, дальність і час польоту снаряда.

\vspace{1em}
\textit{Варіант 9}\par
Тіло кинуте під кутом 45 градусів до поверхні землі. Визначити дальність польоту, якщо початкова швидкість дорівнює 60 м/с, маса тіла – 3 кг, коефіцієнт тертя – 0,1.
