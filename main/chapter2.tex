%!TEX root = ../bdr.tex

\chapter{Створення бібліоґрафії}

У TeХ \emph{бібліографія створюється автоматично}. Для цього у необхідному місці тексту достатньо вставити команду \verb|\bibliography{bibdata}| і аргументом вказати ім'я файлу із літературними джерелами.
Цей файл повинен мати розширення \verb|*.bib| (наприклад, для цього шаблону використовується файл \emph{bibdata.bib}). 

У цьому файлі знаходиться \emph{простий невпорядкований список} записів кожен з яких
описує рівно одну публікацію (тип видання, назва, автор, рік, сторінки тощо). Редагувати цей файл можна або спеціальною програмою (наприклад, JabRef) - і це зручно, або навіть будь-яким текстовим редактором. 

Власне оформлення самої бібліографії здійснюється повністю автоматично. При цьому відбувається впорядкування, нумерація і форматувння літературних джерел згідно правил. 

У бібліографію потрапляють не всі літературні джерела із файлу \verb|*.bib|, а лише ті, на які було здійснено посилання із тексту. Це означає, що можна вести один універсальний файл (базу даних) з літературою за певною тематикою, а у статтях посилатись лише на необхідні.

\section{Посилання на літературу}

Якщо існує bib-файл і він підключений до команди ``Бібліографія'', то у тексті можна робити посилання на літературні джерела, наприклад, ось так \cite{WinNT} або так \cite{Vasylenko92, Afanasyev92} або так \cite{Makilov91, Ponomarenko86, Belousova81, Tezisy, Statia, GOST7184}.

Нумерація літератури у тексті і у бібліографії здійснюється автоматично. 

