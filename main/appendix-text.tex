%!TEX root = ../bdr.tex
\chapter[(Довідковий) Приклад текстового додатку]{(Довідковий)\\ Приклад текстового додатку}
\label{apdx:text}

Додатки створюються так само, як і основний текст пояснювальної записки. Будь-який додаток -- це окремий розділ, який
починається командою \verb|\chapter{}| але, на відміну від розділів, що розташовуються в основній частині роботи, заголовок
додатку буде формуватись зі слова <<Додаток>>  і літери алфавіту розташованих згори по центру автоматично. Таку зміну
стандартної поведінки команди \verb|\chapter{}| провокує команда \verb|\appendix|, яка розбиває документ на дві частини: 
основну частину та частину з додатками.


Як і в основній частині роботи в додатках можна розташовувати рисунки, проте нумеруватись вони будуть в межах додатку
починаючись літерою, як показано на рисунку \ref{apdxfig:tux}. Посилання на рисунки розташовані в додатках можна розташовувати 
в будь-якому місці документу (дивись сторінку \pageref{linkpage}).


\begin{figure}[h]
 \centering\includegraphics{img/Tux.png}
 \caption{Пінгвін}
 \label{apdxfig:tux}
\end{figure}

Все вище описане можна застосувати і для таблиць. Як і рисунки таблиці будуть нумеруватись в межах додатку починаючи літерою.
Посилатись на таблиці розташовані в додатку можна так само як і на будь-які інші таблиці в документі. 

\begin{table}[h]
\caption{\label{apdxtable:1}Функціональна залежність параметрів ...}
 \begin{tabular}{|c|c|c|}
 \hline 
 Індекс & Показник 1 & Показник 2\tabularnewline
 \hline\hline 
 1      & 2          & 3         \tabularnewline
 \hline
 4      & 5          & 6         \tabularnewline
 \hline 
\end{tabular}
\end{table}
  


\begin{table}[h]
\underonespace
\caption{\label{apdxtable:2}Основні та деякі похідні одиниці системи SI}
\begin{tabular}{|>{\raggedright}m{5cm}|c|m{4cm}|c|}
\hline
\multicolumn{2}{|c|}{Величина}&\multicolumn{2}{c|}{Одиниця}\\[0pt]\hline\vspace{4pt}
Найменування & Розмірність & Найменування & Позначення \\[0pt]\hline\hline\vspace{4pt}
Довжина & $L$ & метр & м \\[0pt]\hline\vspace{4pt}
Маса & $M$ & кілограм & кг \\[0pt]\hline\vspace{4pt}
Час & $T$ & секунда & с \\[0pt]\hline\vspace{4pt}
Сила електричного струму & $I$ & ампер & А \\[0pt]\hline\vspace{4pt}
Термодинамічна температура & $\Theta$ & кельвін & К \\[0pt]\hline\vspace{4pt}
Кількість речовини & $N$ & моль & моль \\[0pt]\hline\vspace{4pt}
Сила світла & $J$ & кандела & кд \\[0pt]\hline
\hline\vspace{4pt}
Плаский кут & $1$ & радіан & рад \\[0pt]\hline\vspace{4pt}
Площа & $L^2$ & квадратний метр & м$^2$ \\[0pt]\hline\vspace{4pt}
Об'єм & $L^3$ & кубічний метр & м$^2$\\[0pt]\hline\vspace{4pt}
Швидкість & $LT^{-1}$ & метр за секунду & м/c \\[0pt]\hline\vspace{4pt}
Прискорення & $LT^{-2}$ & метр за секунду в квадраті & м/с$^2$\\[0pt]\hline\vspace{4pt}
Кутова швидкість & $T^{-1}$ & радіан за секунду & рад/с \\[0pt]\hline\vspace{4pt}
Кутове прискорення & $T^{-2}$ & радіан за секунду в квадраті & рад/с$^2$ \\[0pt]\hline\vspace{4pt}
Щільність & $ML^{-3}$ & кілограм на метр кубічний & кг/м$^3$ \\[0pt]\hline\vspace{4pt}
Питомий об'єм & $L^3M^{-1}$ & кубічний метр на кілограм & м$^3$/кг \\[0pt]\hline\vspace{4pt}
Сила & $LMT^{-2}$ & ньютон & Н \\[0pt]\hline\vspace{4pt}
Тиск & $L^{-1}MT^{-2}$ & паскаль & Па \\[0pt]\hline\vspace{4pt}
Температура Цельсія & $\Theta$ & градус Цельсія & $^\circ$C \\[0pt]\hline 
\end{tabular}
\end{table}