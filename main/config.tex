%!TEX root = ../bdr.tex
% основні параметри які можна змінити/задати в документі класу bdrvntu

%% навчальний заклад. По замовчанню "Вінницький національний технічний унверситет", але можна змінити на інший командою \educational{}, як наприклад:
%\educational{Національний університет <<Львівська політехніка>>}
%% навчальний заклад абревіатурою. По замовчанню "ВНТУ", але можна змінити на іншу командою \educationalabbr{}, як наприклад:
%\educationalabbr{НУЛП}
%% місто. По замовчанню "Вінниця", але можна змінити на інше командою \city{}, як наприклад:
%\city{Львів}
%% факультет. По замовчанню "комп'терних систем і автоматики", але можна змінити на інший командою \faculty{}, як наприклад:
\faculty{інтелектуальних інформаційних технологій та автоматизації}
%% кафедра. По замовчанню "автоматизації та інтелектуальних інформаційних технологій", але можна змінити на іншу командою \department{}, як наприклад:
%\department{метрології та промислової автоматики}
%% кафедра абревіатурою. По замовчанню "АІІТ", але можна змінити на іншу командою \departmentabbr{}, як наприклад:
%\departmentabbr{МПА}
%% тема роботи. Значення по замочанню не присвоєне, необхідно обов'язково задавати.
\title{вдосконалення системи управління якості програмного забезпечення}

%% студент
%% курсу. Значення по замочанню не присвоєне, необхідно обов'язково задавати.
\course{IV}
%% групи. Значення по замочанню не присвоєне, необхідно обов'язково задавати.
\group{АКІТ-20б}
%% спеціальності. Значення по замочанню не присвоєне, необхідно обов'язково задавати.
\speciality{151 -- Автоматизація та комп’ютерно-інтегровані технології}
%% ім'я. Значення по замочанню не присвоєне, необхідно обов'язково задавати.
\author{Петровський Д. Ю.}
%% ім'я в родовому відмінку. Фігурує в індивідуальному завданні. Значення по замочанню не присвоєне, необхідно обов'язково задавати.
\gcauthor{Петровському Дмитру Юрійовичу}
     
%% керівник
%% ім'я. Значення по замочанню не присвоєне, необхідно обов'язково задавати.
\leader{Овчинников К. В.}
%% науковий ступінь керівника. Значення по замочанню не присвоєне, необхідно обов'язково задавати.
\degree{к.т.н.}
%% педагогічне звання. Значення по замочанню не присвоєне, необхідно обов'язково задавати.
\position{доцент}
%% передбачається, що керівник з кафедри здобувача. Абревіатура для них однакова і задається командою \departmentabbr{}, що описани вище.

%% рецензент
%% ім'я. Значення по замочанню не присвоєне, необхідно обов'язково задавати.
\reviewer{Тарновський М. Г.}
% науковий ступінь рецензента. Значення по замочанню не присвоєне, необхідно обов'язково задавати.
\reviewerdegree{к.т.н.}
%% педагогічне звання. Значення по замочанню не присвоєне, необхідно обов'язково задавати.
\reviewerposition{доцент}
%% кафедра рецензента абревіатурою. Значення по замочанню не присвоєне, необхідно обов'язково задавати. 
\rdepartmentabbr{КСУ}

%% рік видання
%% рік видання проставляється автоматично в момент останньої компіляції, але якщо потрібно його можна змінити командою \annum{}, як наприклад:
% \annum{2021}

%% індивідуальне завдання
%% галузь знань. Значення по замочанню не присвоєне, необхідно обов'язково задавати.
\branchofknowledge{автоматизація та приладобудування} 
%% освітня програма. Значення по замочанню не присвоєне, необхідно обов'язково задавати.
\educationalprogram{автоматизація та комп’ютерно-інтегровані технології}

%% для креслеників
%% перевірив (нормоконтроль). Значення по замочанню не присвоєне, необхідно обов'язково задавати.
\controller{Овчинников К. В.}
%% затвердив. Значення по замочанню не присвоєне, необхідно обов'язково задавати.
\approver{Бісікало О. В.} 

%% статус очільника кафедри. Значення по замовчанню "Завідувач кафедри"
\dpheaderstatus{в.о. зав. кафедри}

\typeofwork{Бакалаврська дипломна робота}
\acstypeofwork{бакалаврську дипломну роботу}







