%!TEX root = ../bdr.tex
\chapter*{Вступ}

Пояснювальна записка до бакалаврської кваліфікаційної роботи є\linebreak складним та великим за обсягом документом, який містить
багато iлюстрацiй, таблиць, математичних формул, посилань на структурнi частини роботи, формули та джерела у списку 
лiтератури. Протягом роботи над пояснювальною запискою автор повинен постiйно редагувати свiй генiальний текст,
часто термiново. В таких умовах важко витримувати сталою загальну структуру роботи, слідкувати за тим, щоб посилання
на всі структурні частини тексту, рисунки, таблиці, літературні джерела, тощо залишались коректними. І це далеко не все,
що може <<поїхати>> в тексті при редагуванні. Допомогти в цьому може система \LaTeX, одна з найпотужнiших i найефективнiших 
сучасних систем підготовки документів, що ґрунтується на системi комп’ютерного верстування \TeX \cite{latexctan}.

Клас {\LaTeX} {\verb|bdrvntu|} призначений для оформлення пояснювальної записки до бакалаврської кваліфікаційої роботи згiдно вимог,
що висуваються до такого роду робіт у Вінницькому національному технічному університеті а саме:

\begin{itemize}
\item оформлення титульної сторінки;
\item оформлення сторінки індивідуального завдання;
\item оформлення заголовків, розділів, підрозділів та додатків;
\item нумерація сторінок, рисунків, таблиць, формул, тощо;
\item оформлення списку використаних джерел та ін.
\end{itemize} 

Як i будь-який клас документа, він має допомогти автору роботи зосередитися на написаннi власне тексту i використовувати 
логiчну розмiтку тексту замiсть його безпосереднього оформлення.
